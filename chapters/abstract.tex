% !TeX root = ../main.tex

\ustcsetup{
  keywords = {
    冷原子物理,玻色型BdG系统,拓扑能带理论,Bogoliubov理论,Slave Boson,Higgs玻色子,玻色超流
  },
  keywords* = {
    Cold Atom Physics, Bosonic BdG System, Topological Band Theory, Bogoliubov Theory, Slave Boson, Higgs Boson, Bose Superfluid
  },
}

\begin{abstract}

自1980年量子Hall效应被发现开始,拓扑在凝聚态物理中扮演着中心角色。
用拓扑表征的物态打破了传统的Ginzburg-Landau理论,
即不同的物质态可以有相同的对称性和对称性破缺行为,但是它们被不同的拓扑属性区分。
在拓扑物质态这一大类新型物态中,拓扑绝缘体和拓扑超导体属于自由费米子对称保护拓扑态。
其理论基础是拓扑能带理论。
这类拓扑物态的一个重要特征是它们有无能隙的边界态;
特别地,人们期望在拓扑超导体中找到Majorana零能模,并用来进行量子计算。然而正如最早Haldane在2005年指出的那样,拓扑能带理论不是费米子独有的,它本质上是一种波的效应。
随后人们也开始研究其他物理体系中类似的拓扑现象。
事实上,对于各种玻色型量子多体体系,在它们的有序相,其集体激发也有能带结构,例如晶格中的声子或顺磁相中的磁子。
这些能带结构也可以是拓扑的,它们也有对应的物理效应(如热Hall传导)和表面态。

本文研究的是玻色超流体集体激发的能带拓扑行为,这些理论可以直接应用于光晶格中的超冷玻色子体系。
具体地,在第一部分,我们讨论了赝时间反演对称性保护的玻色超流体Bogoliubov激发谱。
由于玻色型BdG哈密顿量的对角化是通过一个赝幺正矩阵完成的,通常的拓扑能带理论并不直接适用;
各种对称性保护的拓扑性质也需要重新考察。
这里我们研究了一类最重要的对称性保护拓扑能带——时间反演不变的二维拓扑绝缘体(AII类)——在玻色超流体Bogoliubov激发谱中的对应。
我们首先给出了赝时间反演对称玻色型BdG哈密顿量的三种等价体拓扑数$\mathbb Z_2$不变量的定义。然后利用Bogoliubov理论,具体计算了两个玩具模型,Kane-Mele-Bose-Hubbard模型和Bernevig-Hughes-Zhang-Bose-Hubbard模型,在弱相互作用极限超流相的激发谱,通过考察其对应的体拓扑,我们发现相比于单粒子能带,拓扑区间随着相互作用强度增加而变大,这可解释为排斥相互作用倾向于一个均匀的构型从而实际上压制了调节拓扑相变的参数。通过具体计算开放边界条件下的激发谱,我们也发现了一对螺旋的边界态,数值地验证了体边对应关系。

在第二部分,我们把注意力转向了强相互作用区间,考察了$d=2$维的Su-Schrieffer-Heeger-Bose-Hubbard模型在超流相到Mott绝缘相转变临界点附近的激发谱。这里有一个让人着迷的问题:量子相变点附近可能有涌现的拓扑Higgs振幅模吗?我们从三个方面回答了这个问题。首先在大整数填充极限下,通过把该模型映射到一个赝自旋-1模型,我们计算了后者在自发对称破缺相变临界点附近的激发谱,发现有完整的Higgs振幅能带存在;并且其拓扑行为直接继承于单粒子能带。接着,利用slave boson方法,我们计算了在有限整数填充下原始模型在临界点附近的激发谱。一方面,我们指出在有限填充下这个激发谱的拓扑性质不会发生变化;另一方面,通过定义平坦度$F$来定量刻画能带的振幅-相位特性,我们发现在实验允许的填充数下,已经有涌现的Higgs振幅拓扑边界态。最后,利用强耦合RPA方法,我们计算了以振幅和相位作为变量的低能有效作用量,通过一个类似Ginzburg-Landau理论的分析,互补、统一地解释了上述结果。




\end{abstract}

\begin{abstract*}  
Since the discovery of quantum Hall effect in 1980, topology has been playing a central role in condensed matter physics. The usage of topology to characterize phases of matter is beyond the paradigm of Ginzburg-Landau theory, i.e., different phases of matter may possess same symmetry and symmetry-breaking pattern, but they are distinct due to topological attribute. In the huge family of topological phases of matter, topological insulators and topological superconductors belong to the so-called free fermion symmetry-protected topological states. The underlying theoretical tool for them is topological band theory. One of the important characteristics of topological matter is the appearance of gapless edge modes; in particular, Majorana zero modes are expected to be found in topological superconductors, which can be used to perform quantum computing. However, as Haldane firstly pointed out in 2005, topological band theory is not tied to fermions, but essentially a wave effect. Since then there are trends in studying similar topological phenomena in other physical systems. In fact,
  for various kinds of bosonic quantum many-body systems, at their ordered phase, the collective excitations also form energy bands, e.g., phonons in crystals and magnons in magnets. And these bands can also be topological, which has their own physical consequences (e.g., thermal Hall effect) and surface states.

  In this thesis we study symmetry-protected topological excitations of Bose superfluids, which can be applied directly to cold atom systems in optical lattices. Specifically, in the first part, we discuss pseudo-time-reversal-symmetry-protected topological Bogoliubov excitations of Bose superfluids. Because bosonic BdG Hamiltonian is diagonalized pseudounitarily, the conventional topological band theory is not applicable,
  and various kinds of symmetry-protected topological properties must be reexamined. Here we identify the counterpart of the most important symmetry-protected topological bands --- time reversal invariant topological insulators in two dimensions (AII class) ---  in the excitation spectrum of a weakly interacting Bose superfluids.
  We first give three equivalent bulk topological $\mathbb Z_2$ invariants for pseudo-time-reversal symmetric bosonic BdG Hamiltonians in two dimensions. Then utilizing Bogoliubov theory, topology of excitation spectrum of two toy models, Kane-Mele-Bose-Hubbard model and Bernevig-Hughes-Zhang-Bose-Hubbard model, are studied in details in the weak-coupling limit. Comparing to their single-particle Hamiltonian, we find the topological region enlarges as the interaction strength increasing, this is because the repulsive interaction favors a uniform configuration, which suppresses effects of topological-phase-transition tuning parameter. We also find a pair of helical edges modes under open boundary conditions in the topological region, which verifies the bulk-boundary correspondence numerically.
  
  In the second part, we turn our attentions to the strong coupling regions, and examine the excitation spectrum of the $d$ dimensional Su-Schrieffer-Heeger-Bose-Hubbard model near the superfluid-Mott insulating phase transition boundary. There is an intriguing question: Are there emergent \emph{topological} Higgs amplitude modes near the quantum phase transition point? We give a confirmative answer from three aspects: Firstly, in the large integer filling limit, via mapping it to a pseudo-spin-1 model, we calculate the excitation spectrum of the latter near the phase transition point. We find the emergent pure Higgs amplitude bands, whose topological properties are directly inherited from the underlying single-particle Hamiltonian. Then using a slave boson approach, we calculate the excitation spectrum of the original model at a finite integer filling near the phase transition point. On the one hand, it is argued that the topological properties of these excitations are not altered from the previous case; on the other hand, by defining a flatness parameter to quantitatively characterize the amplitude-phase character of these modes, we found numerically that at an experimentally feasible filling, there are emergent Higgs amplitude topological edge modes when open boundary conditions are imposed. Finally, applying a strong-coupling RPA, we obtain the low-energy effective action in terms of amplitude and phase variables; a Ginzburg-Landau-like analysis based on this action complementarily explains all above results in a unified way.
  

























































\end{abstract*}
