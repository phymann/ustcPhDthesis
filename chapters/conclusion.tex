% !TeX root = ../main.tex

\chapter{总结和展望}

\paragraph{全文总结}

本文研究光晶格中超冷玻色子在超流相的激发谱能带的拓扑性质。第一章中我们首先概述了整个拓扑物态的发展历史脉络,然后具体讨论了规范理论中拓扑的应用,最后详细介绍了传统(自由费米子)拓扑能带理论。在第二章中,我们首先介绍了Bogoliubov变换的定义,然后具体讨论了如何求解一个一般二次型哈密顿量(包括玻色型和费米型),最后给出了一个数值算法。第三章研究了怎么把对称性保护拓扑能带理论中最经典的例子,二维时间反演对称性保护的拓扑能带(AII类),推广到玻色超流体的激发谱能带中。这里我们首先介绍了Krein空间的定义,然后定义了赝时间反演不变性,并且定义了三种(等价)体拓扑数;接着具体计算了两个玩具模型,发现其激发谱确有类似AII类的拓扑能带,还数值确认了体边对应关系。最后一章我们研究在强相互作用量子相变点附近激发谱的能带拓扑。这里我们关心的问题是:量子相变点附近能否有涌现的拓扑Higgs振幅模。为了回答这个问题,我们分别考虑了大整数填充极限和有限整数填充的情况。结论表明Higgs振幅拓扑边界态是存在的。最后我们还利用强耦合RPA方法分析了低能有效作用量,互补、统一地解释了前述结果。

\paragraph{展望}

在研究超冷玻色子激发谱的拓扑能带中,一个最重要的问题是怎么进行有效的实验测量,即探测激发谱能带是否有预期的拓扑结构。在费米子拓扑绝缘体里,这些拓扑边界态穿过费米能级,可以通过无穷小的能量激发;具体到费米子冷原子实验上,也有一些通过淬火动力学的方案实现测量的漂亮工作\cite{Wang2017,Zhang2018,Sun2018a,Zhang2019}。为了观测本文中激发谱能带间拓扑边界态,一个高频的探测是必须的。Kurukawa和Ueda\cite{Furukawa2015}提出通过受激Raman跃迁,选择性地把一部分凝聚体转移到特定动量和频率的对应边界态,实现所谓的“边界物质波”。这种局域在边界的传播波可以作为一个宏观观察现象来作为拓扑边界态存在的证据,类似计算也出现在文献\cite{Xu2016}。然而这个方法还不够直接;我们期望在之后的工作中能构思出更加有效、决定性的方案实现激发谱拓扑能带的探测。

理论上,也有一些非常有趣的问题值得进一步探索。例如寻找那些在玻色型BdG系统能带中特有的拓扑结构。具体来说,这联系到非厄米哈密顿量区别与厄米情况的、特有拓扑性质,这方面已有一些讨论\cite{Lieu2018,Kawabata2018},但还不够完整\cite{Lein2019}。另外,寻找在冷原子、磁性材料等玻色型BdG系统中独特的拓扑能带实现机制。例如,区别于本文中的激发谱能带拓扑继承于单粒子能带,利用BdG哈密顿量中的反常(粒子数不守恒)项实现能带拓扑\cite{McClarty2018},这种情况被文献\cite{Wan2021}称作所谓的“压缩拓扑态”。