% !TeX root = ./main.tex

\ustcsetup{
  title              = {光晶格中弱、强相互作用超流的拓扑激发模},
  title*             = {An example of thesis template for University of Science
                        and Technology of China},
  author             = {王俊森},
  author*            = {Wang, Junsen},
  speciality         = {量子物理学},
  speciality*        = {quantum physics},
  supervisor         = {邓友金~教授},
  supervisor*        = {Prof. Youjin Deng},
  % date               = {2017-05-01},  % 默认为今日
  % professional-type  = {专业学位类型},
  % professional-type* = {Professional degree type},
  % secret-level       = {秘密},     % 绝密|机密|秘密,注释本行则不保密
  % secret-level*      = {Secret},  % Top secret|Highly secret|Secret
  % secret-year        = {10},      % 保密年限
  %
  % 数学字体
  % math-style         = GB,  % 可选:GB, TeX, ISO
  math-font          = xits,  % 可选:stix, xits, libertinus
}


% 加载宏包

% 定理类环境宏包
\usepackage{amsthm}

\usepackage{physics}
%\usepackage{mathtools}

% 插图
\usepackage{graphicx}

% 三线表
\usepackage{booktabs}

% 跨页表格
\usepackage{longtable}

% 算法
\usepackage[ruled,linesnumbered]{algorithm2e}

% SI 量和单位
\usepackage{siunitx}

% 参考文献使用 BibTeX + natbib 宏包
% 顺序编码制
\usepackage[sort]{natbib}
\bibliographystyle{ustcthesis-numerical}

% 著者-出版年制
% \usepackage{natbib}
% \bibliographystyle{ustcthesis-authoryear}

% 本科生参考文献的著录格式
% \usepackage[sort]{natbib}
% \bibliographystyle{ustcthesis-bachelor}

% 参考文献使用 BibLaTeX 宏包
% \usepackage[style=ustcthesis-numeric]{biblatex}
% \usepackage[bibstyle=ustcthesis-numeric,citestyle=ustcthesis-inline]{biblatex}
% \usepackage[style=ustcthesis-authoryear]{biblatex}
% \usepackage[style=ustcthesis-bachelor]{biblatex}
% 声明 BibLaTeX 的数据库
% \addbibresource{bib/ustc.bib}

% 配置图片的默认目录
\graphicspath{{figures/}}

% 数学命令
\makeatletter
\newcommand\dif{%  % 微分符号
  \mathop{}\!%
  \ifustc@math@style@TeX
    d%
  \else
    \mathrm{d}%
  \fi
}
\makeatother
\newcommand\eu{{\symup{e}}}
\newcommand\iu{{\symup{i}}}
\newcommand\hc{{\text{h.c.}}}
\newcommand\bdg{{\text{BdG}}}
\newcommand\eff{{\text{eff}}}
\newcommand{\krin}[1]{\langle\!\langle #1 \rangle\!\rangle}
\newcommand{\inv}{^{-1}}
\newcommand{\pf}{\textnormal{Pf}}
\newcommand{\tmax}{{\textnormal{max}}}
\newcommand{\tgp}{\textnormal{GP}}
\newcommand{\bog}{\textnormal{Bog.}}
\newcommand{\sgn}{\textnormal{sgn}}
\newcommand{\tint}{\textnormal{int}}

% 用于写文档的命令
\DeclareRobustCommand\cs[1]{\texttt{\char`\\#1}}
\DeclareRobustCommand\pkg{\textsf}
\DeclareRobustCommand\file{\nolinkurl}

% hyperref 宏包在最后调用
\usepackage{url}
\usepackage{hyperref}
\hypersetup{
	bookmarksnumbered,		% (sub)section numbers in bookmarks
	unicode,			% use with \texorpdfstring
	colorlinks,			% avoid stupid boxes
	citecolor=[rgb]{.9,0,.5},	% \cite
	urlcolor=[rgb]{0,0,1},		% \href
	linkcolor=[rgb]{0,.7,0}		% \ref , toc
	}

